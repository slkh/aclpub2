\noindent{\large \textbf{January 21, 2025}}\\  \vspace*{-0.1cm} \leavevmode\newline

\noindent{Session 1: \textbf{Arabic NLP}} \\
\noindent{Organizer: \textbf{Nizar Habash}}\\
\noindent{Time: \textbf{14:00 - 15:30} - Location: \textbf{Hall B Room D}}\\
Description: This Birds-of-a-Feather session is designed for researchers and developers working on or interested in Arabic NLP. Organized by a team affiliated with SIGARAB, the ACL Special Interest Group on Arabic Natural Language Processing, the session focuses on social networking and engaging discussions on timely and relevant topics.\\

\noindent{Session 2: \textbf{NLP for Social Good}}\\
\noindent{Organizer: \textbf{Daryna Dementieva}}\\
\noindent{Time: \textbf{16:00 - 17:30} - Location: \textbf{Hall B Room D}}\\
Description: While LLMs and other NLP/AI advancements present us with tremendous opportunities, they also demand greater responsibility from us as NLP researchers. One key responsibility is to bridge the gap between technology and real-world societal challenges, considering how AI/NLP can effectively address them. At our BoF session, we will discuss pressing issues---i.e. digital violence, fake news and propaganda detection (and feel free to propose your own topic). We will discuss both successful and less effective cases of NLP application to understand what drives these outcomes. Our aim is to create a space for researchers worldwide to connect and collectively envision a brighter future, using NLP for the greater good.\\

\noindent{\large \textbf{January 22, 2025}}\\  \vspace*{-0.1cm} \leavevmode\newline

\noindent{Session 3: \textbf{Language Technology for Crisis Preparedness and Response }} \\
\noindent{Organizer: \textbf{Belu Ticona}}\\
\noindent{Time: \textbf{11:00 - 12:30} - Location: \textbf{Hall B Room D}}\\
Description: How can we use language technology to prepare for the next crisis? In a world with more natural disasters and socio-political conflicts, communication is crucial so that victims can share their needs (in their languages/dialects), organizations can provide humanitarian aid, and broadcast urgent information. In this BoF, we will discuss challenges and possible solutions through a role-playing activity. We invite anyone interested in low-resource languages, machine translation, multicultural NLP, climate change, or crisis and language in general.\\

\noindent{Session 4: \textbf{Southeast Asian NLP}}\\
\noindent{Organizer: \textbf{Jan Christian Blaise Cruz}}\\
\noindent{Time: \textbf{14:00 - 15:30} - Location: \textbf{Hall B Room D}}\\
Description: Small group discussions and community-driven talks around the current challenges, directions, and opportunities in Southeast Asian NLP. The BoF is organized by co-founders of the ACL SIGSEA and SEACrowd. We aim to host one BoF session in every major NLP conference after the success of the ones hosted in ACL 2024 and EMNLP 2024.\\

\noindent{\large \textbf{January 23, 2025}}\\  \vspace*{-0.1cm} \leavevmode\newline

\noindent{Session 5: \textbf{Supporting Mental Health Through NLP-Powered Conversational Agents}} \\
\noindent{Organizer: \textbf{Sadegh Jafari}}\\
\noindent{Time: \textbf{09:00 - 10:30} - Location: \textbf{Hall B Room D}}\\
Description: We will explore how advanced language models and classification techniques can detect stress disorders, analyze emotions, and provide personalized assistance to users. Attendees will have the opportunity to discuss challenges, ethical considerations, and future directions for using NLP in mental health, fostering collaboration between researchers, practitioners, and technologists passionate about making a positive impact on mental well-being.\\

\noindent{Session 6: \textbf{Multilingual and Multimodal Cultural Inclusivity in LLMs}}\\
\noindent{Organizer: \textbf{Firoj Alam}}\\
\noindent{Time: \textbf{11:00 - 12:30} - Location: \textbf{Hall B Room D}}\\
Description: The Birds of a Feather session on "Multilingual and Multimodal Cultural Inclusivity in LLMs" aims to foster a collaborative dialogue on the challenges and opportunities of creating LLMs that are inclusive of diverse languages, cultures, and modalities. As LLMs increasingly influence global communication and understanding, ensuring they represent and respect cultural nuances, underrepresented languages, and multimodal interactions is critical. This session will explore current advancements, challenges in addressing linguistic and cultural biases, and strategies for incorporating multilingual and multimodal datasets into LLM training. Participants will discuss the ethical, technical, and societal implications of these efforts, sharing ideas and solutions to make AI more inclusive and equitable across diverse linguistic and cultural landscapes.\\

\noindent{\large \textbf{January 24, 2025}}\\  \vspace*{-0.1cm} \leavevmode\newline

\noindent{Session 7: \textbf{Exploring Public Sentiment on AI}} \\
\noindent{Organizer: \textbf{Rushil Thareja}}\\
\noindent{Time: \textbf{10:30 - 12:00} - Location: \textbf{Hall B Room D}}\\
Description: With the widespread adoption of AI across various aspects of life, productivity has increased, and many find life simpler and more convenient. Yet, a growing segment of society views AI with skepticism or negativity, often perceiving it as unnecessary, harmful, or driven by capitalist motives. This sentiment raises a critical question: if AI is meant to improve lives, why do so many feel otherwise? The goal of this session is for AI researchers to come together to understand these perceptions, challenge assumptions about their root causes, and rethink how AI is designed and implemented. By aligning AI with people's needs and values, we can ensure it becomes a truly positive force that works hand in hand with society.\\