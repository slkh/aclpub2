\chapter{Tutorials Details}
\subsection{Sunday, January 19, 2025}
\noindent{\textbf{Tutorial 1: Speculative Decoding for Efficient LLM Inference}}\\  
\noindent{\emph{Organizers: Heming Xia, Cunxiao Du, Yongqi Li, Qian Liu and Wenjie Li}}\\
 
\noindent{This tutorial presents a comprehensive introduction to Speculative Decoding (SD), an advanced technique for LLM inference acceleration that has garnered significant research interest in recent years. SD is introduced as an innovative decoding paradigm to mitigate the high inference latency stemming from autoregressive decoding in LLMs. At each decoding step, SD efficiently drafts several future tokens and then verifies them in parallel. This approach, unlike traditional autoregressive decoding, facilitates the simultaneous decoding of multiple tokens per step, thereby achieving promising 2x-4x speedups in LLM inference while maintaining original distributions. This tutorial delves into the latest techniques in SD, including draft model architectures and verification strategies. Additionally, it explores the acceleration potential and future research directions in this promising field. We aim for this tutorial to elucidate the current research landscape and offer insights for researchers interested in Speculative Decoding, ultimately contributing to more efficient LLM inference.} \\

\noindent{Date: Jan. 19, 2025}\\
\noindent{Time: 09:00-12:30}\\
\noindent{Location: Capital Suite 7}\\

\noindent\hrulefill
 
\noindent{\textbf{Tutorial 2: From Theory to Practice: A Hands-on Tutorial in Explainable NLP}}\\  
\noindent{\emph{Organizers: Wafa Abdullah Alrajhi, Nourah Alangari, and Hend Al-Khalifa}}\\

\noindent{This tutorial aims to explore the intersection between XAI and the field of Natural Language Processing (NLP) by highlighting its significance for various tasks. By combining both theoretical and practical aspects of XAI, we provide a comprehensive overview of its definitions and review the state-of-the-art methodologies employed to explain NLP models. While previous tutorials have focused on establishing the theoretical foundations of the XAI field and providing essential concepts to the audience, this tutorial seeks to bridge the gap between theory and practice. We will discuss and demonstrate the most commonly used XAI techniques, the prominent libraries applicable to different models, and the limitations of each method. Furthermore, we will highlight the methods that can be utilized to visualize these models while performing specific NLP tasks.}\\

\noindent{Date: Jan. 19, 2025}\\
\noindent{Time: 09:00-12:30}\\
\noindent{Location: Online Only}\\

\noindent\hrulefill
 
\noindent{\textbf{Tutorial 3: Hands-On Tutorial: Labeling with LLM and Human-in-the-Loop}}\\  
\noindent{\emph{Organizers: Ekaterina Artemova, Akim Tsvigun, Dominik Schlechtweg, Natalia Fedorova, Sergei Tilga, Boris Obmoroshev and Konstantin Chernyshev}}\\

\noindent{Training and deploying machine learning models relies on a large amount of human-annotated data. As human labeling becomes increasingly expensive and time-consuming, recent research has developed multiple strategies to speed up annotation and reduce costs and human workload: generating synthetic training data, active learning, and hybrid labeling. This tutorial is oriented toward practical applications: we will present the basics of each strategy, highlight their benefits and limitations, and discuss in detail real-life case studies. Additionally, we will walk through best practices for managing human annotators and controlling the quality of the final dataset. The tutorial includes a hands-on workshop, where attendees will be guided in implementing a hybrid annotation setup. This tutorial is designed for NLP practitioners from both research and industry backgrounds who are involved in or interested in optimizing data labeling projects.}\\

\noindent{Date: Jan. 19, 2025}\\
\noindent{Time: 14:00-17:30}\\
\noindent{Location: Capital Suite 5}\\

\noindent\hrulefill

\noindent{\textbf{Tutorial 4: LLMs in Education: Novel Perspectives, Challenges, and Opportunities}}\\  
\noindent{\emph{Organizers: Bashar Alhafni, Sowmya Vajjala, Stefano Banno, Kaushal Kumar Maurya and Ekaterina Kochmar}}\\

\noindent{The role of large language models (LLMs) in education is an increasing area of interest today, considering the new opportunities they offer for teaching, learning, and assessment. This cutting-edge tutorial provides an overview of the educational applications of NLP and the impact that the recent advances in LLMs have had on this field. We will discuss the key challenges and opportunities presented by LLMs, grounding them in the context of four major educational applications: reading, writing, and speaking skills, and intelligent tutoring systems (ITS). This tutorial is designed for researchers and practitioners interested in the educational applications of NLP and the role LLMs have to play in this area. It is the first of its kind to address this timely topic.}\\

\noindent{Date: Jan. 19, 2025}\\
\noindent{Time: 14:00-17:30}\\
\noindent{Location: Capital Suite 7}\\

\noindent\hrulefill

\noindent{\textbf{Tutorial 5: EduRAG: Crafting Clever Educational Chatbots and Advanced QA Systems with Retrieval-Augmented Generation}}\\  
\noindent{\emph{Organizers: Noorhan Abbas and Saad Ezzini}}\\

\noindent{Date: Jan. 19, 2025}\\
\noindent{Time: 14:00-17:30}\\
\noindent{Location: Online Only}\\


\subsection{Monday, January 20, 2025}
\noindent{\textbf{Tutorial 6: Connecting Ideas in Lower-Resource Scenarios: NLP for National Varieties, Creoles, and Other Low-Resource Scenarios}}\\  
\noindent{\emph{Organizers: Aditya Joshi, Diptesh Kanojia, Heather Lent, Hour Kaing and Haiyue Song}}\\

\noindent{Despite excellent results on benchmarks over a small subset of languages, large language models struggle to process text from languages situated in ‘lower-resource’ scenarios such as dialects/sociolects (national or social varieties of a language), Creoles (languages arising from linguistic contact between multiple languages) and other low-resource languages. This introductory tutorial will identify common challenges, approaches, and themes in natural language processing (NLP) research for confronting and overcoming the obstacles inherent to data-poor contexts. By connecting past ideas to the present field, this tutorial aims to ignite collaboration and cross-pollination between researchers working in these scenarios. Our notion of ‘lower-resource’ broadly denotes the outstanding lack of data required for model training - and may be applied to scenarios apart from the three covered in the tutorial.}\\

\noindent{Date: Jan. 20, 2025}\\
\noindent{Time: 09:00 - 17:30}\\
\noindent{Location: Conference Hall B (C)}\\

\noindent\hrulefill

\noindent{\textbf{Tutorial 7: Safety Issues for Generative AI}}\\  
\noindent{\emph{Organizers: Haonan Li, Xudong Han, Emad A. Alghamdi, Shom Lin, Monojit Choudhury, Jingfeng Zhang, Paul Rottger and Timothy Baldwin}}\\

\noindent{This tutorial will provide an in-depth exploration of safety issues for generative AI models, covering a broad range of sub-topics, including a risk taxonomy, adversarial attack types, safety evaluation, defense mechanisms, red teaming for multi-modal models, and agentic AI. The goal is to brief attendees on the latest advancements and emerging trends in the field, enabling them to identify and mitigate vulnerabilities in generative AI systems. Participants will gain insights into recent research developments, open research questions, and novel research directions. This cutting-edge tutorial is designed for AI researchers, developers, and security professionals with a basic knowledge of red teaming and AI safety.}\\

\noindent{Date: Jan. 20, 2025}\\
\noindent{Time: 09:00-12:30}\\
\noindent{Location: Capital Suite 7}\\

\noindent\hrulefill

\noindent{\textbf{Tutorial 8: Hallucinative Foundation Models: Characterization, Quantification, Avoidance, and Mitigation}}\\  
\noindent{\emph{Organizers: Vipula Rawte, Aman Chadha, Amit Sheth and Amitava Das}}\\

\noindent{In the fast-paced domain of Large Language Models (LLMs), the issue of hallucination is a prominent challenge. Despite continuous endeavors to address this concern, it remains a highly active area of research within the LLM landscape. Grasping the intricacies of this problem can be daunting, especially for those new to the field. This tutorial aims to bridge this knowledge gap by introducing the emerging realm of hallucination in LLMs. It will comprehensively explore the key aspects of hallucination, including benchmarking, detection, and mitigation techniques. Furthermore, we will delve into the specific constraints and shortcomings of current approaches, providing valuable insights to guide future research efforts for participants.}\\

\noindent{Date: Jan. 20, 2025}\\
\noindent{Time: 09:00-12:30}\\
\noindent{Location: Online Only}\\

\noindent\hrulefill

\noindent{\textbf{Tutorial 9: Bridging Linguistic Theory and AI: Usage-Based Learning in Humans and Machines}}\\  
\noindent{\emph{Organizers: Claire Bonial, Harish Tayyar Madabushi, Nikhil Krishnaswamy and James Pustejovsky}}\\

\noindent{Usage-based theories of human language, such as Construction Grammar, have been compelling theoretical lenses through which to view and evaluate what LLMs know and understand of language because of the parallels between usage-based learning and the data-driven ``learning'' of pre-trained models. However, a key difference between a usage-based learning account for humans and that of LLMs is in embodiment and multimodality---for the most part, LLMs use text alone, whereas usage-based theories posit that each token of linguistic experience is stored with a wealth of experiential information that enriches the symbol through cross-modal association. Therefore, the first goal of this tutorial is to provide a summary of language acquisition and second language learning from a usage-based theoretical linguistic perspective. With this understanding of human usage-based learning, we will turn to evidence demonstrating the ways in which machine learning, primarily via large, pre-trained vision and language models, does and does not parallel human learning. The overarching goal of this is not to say that the two processes are similar or dissimilar in order to conclude that dissimilarity denotes inferiority (if the knowledge arrived at is the same, then it may not matter how it was learned). Rather, we explore the resulting differences in what is known and understood about the world, and take this as a starting point for considering how to supplement and improve natural language understanding (NLU), particularly for physically situated applications. Our target audience is those interested in the intersection of linguistic theory and NLU implementations, such as human-robot interaction.}\\

\noindent{Date: Jan. 20, 2025}\\
\noindent{Time: 14:00-17:30}\\
\noindent{Location: Capital Suite 7}\\
     
